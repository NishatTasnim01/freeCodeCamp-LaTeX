\documentclass[11pt]{article}
\usepackage{amsmath, amssymb, amsfonts}

\begin{document}

\section{Superscripts} 

$$2x^3$$
$$2x^{34}$$
$$2x^{3x+4}$$
$$2x^{3x^4+5}$$

\section{Subscripts}
$$x_1$$
$$x_{12}$$
$$x_{1_2}$$
$$x_{1_{2_3}}$$
$$a_0, a_1, a_2, \ldots, a_{100}$$

\section{Greek Letters}
$$\pi$$
$$\Pi$$
$$\alpha$$
$$A=\pi r^2$$

\section{Trigonometric Functions}
$$y=\sin x$$
$$y=\cos x$$
$$y=\csc \theta$$
$$y=\sin^{-1} x$$
$$y=\arcsin x$$

\section{Log Functions}
$$y=\log x$$
$$y=\log_5 x$$
$$y=\ln x$$

\section{Roots}
$$\sqrt{2}$$
$$\sqrt[3]{2}$$
$$\sqrt{x^2+y^2}$$
$$\sqrt{1+\sqrt{x}}$$

\section{Fractions}
$$\frac{2}{3}$$
About $\displaystyle \frac{2}{3}$ of the glass is full.\\[16pt]
About $\frac{2}{3}$ of the glass is full.\\[6pt]
About $\dfrac{2}{3}$ of the glass is full.

$$\frac{\sqrt{x+1}}{\sqrt{x+2}}$$ \\
$$\frac{\sqrt{x+1}}{\sqrt{x+2}}$$ \\
$$\frac{1}{(1+\frac{1}{x})}$$

\end{document}